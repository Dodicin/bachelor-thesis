\chapter{Introduction}

\section{Emergence of the Distributed Ledger model}

Ledgers have been a central element of commerce since ancient times, and are be used to record a variety of informations, from assets to property, but most importantly how these change hands, that is, transactions.

Ledgers have values as \textit{archives}, in other words ledgers have value in their capability of being consulted to check, verify and manage the recorded transactions.

The medium on which transactions have been stored may have changed from clay tablets to hardware storage, but in all this time there haven't been notable innovations to the underlying architecture of the system.
Each financial entity (i.e. banks, governments, financial institutions) manages its own ledgers, with its own technologies and implementative properties, based on their vision, necessities and customers (the would-be \textit{counterparts} in the transaction), and in turn, the counterparts keep recorded their own views of the transactions.

This duplication of information between each party partecipating in the transaction drives a need for costly matching between each copy of the information, reconciliation and error fixing. The plurality of technology platforms upon which financial entities rely adds to that, creating more complexity and operational risks, some of which potentially systematic. \\

Think about the need for a party to transfer an asset, be it cash or a stock, to another. The transaction itself can be executed in microseconds, but the settlement - the ownership transfer of the asset - usually takes more time, from days to weeks.
This length is due to different reasons: the parties don't have access to each other's ledgers, and can't automatically verify that the assets about to be transferred are in fact owned and not fake. So a number of intermediaries are needed as guarantors of the assets and to verify the transaction. 
A number of step has to be added just for this trusting mechanism, and in addition, the difference between infrastructures and technologies of every counterpart acting in the transaction can be such that there's always a need for reconciliation process between each party (ie adjusting each ledger to the transaction), increasing costs and length of the operations.  \\

Centralized infrastructures were until recently an unavoidable model, as there were few ways to consolidate technologies without effectively consolidating the financial entities themselves. The industry has been moving towards the standardization and sharing of data and some of the business-logic behind the architectures through the delegation of some part of the process to third-parties, but these steps are still lagging behind the evolution of the technology. \\

\begin{figure}[t]
    \centering
    \includegraphics[scale=0.3]{architectures-comparison.png}
    \caption{
        Comparison between architectures, \cite{cordawhitepaper}. 
        }
\end{figure}


The term Distributed Ledger Technology refers to the processes, protocols and technologies that enable nodes in a distributed network to share data (propose, validate and record) between multiple synchronized data stores, collectively maintaned. The emergence of this technologies has had a stimulating effect in the FinTech industry, prompting the reconsideration of the entities needed in a financial transaction, how should trust be enstablished, the representation of the transaction, the securing of data, and many more. Whether DLT is not the answer in every case, asking these questions alone can be a force to drive progress forward.

\section{Brief history}
In 2008, a white paper (\cite{bitcoinpaper}) written by an as yet unidentified person using the pseudonym Satoshi Nakamoto, outlined a novel approach of transferring cash from one party to another without the need for a known and trusted third-party in a P2P manner, arguing, amongst other things, to have solved the issue of double-spend for digitalized currencies. Double-spending is the idea that a digital currency can be spent in twice - a unique problem of digital currencies, because digital information can be reproduced relatively easily, unlike physical currency. \\

The technology outlined in the paper was named Blockchain, referring to the way of organizing data and transactions. Bitcoin has soared in terms of popularity and value through its cryptocurrency market, that is, though, just one element in its whole architecture. \\

The effort of the industry since the introduction of blockchains has been directed to exploring different ways of leveraging this technology beyond Bitcoin, focusing on the core architecture of distributed record management. This use has gathered significant attention, reflecting the financial industry traditional reliance on multiple ledgers to maintain transactions. The use of DLT would be particularily effective for payment, clearing and settling activities because of the potential for simplification of the settling and reconciliation process between the parties involved. \\

Some of the resulting implementations of DLTs have been on a steady rise, such as Ethereum, that similarily to Bitcoin has seen a steep rise to the value of its cryptocurrency, Ether, and unlike its predecessor, offers a more malleable enviroment (which is the main reason Vitalik Buterinis created it), allowing for the transfer and recording of other assets like loans or contracts. Other rising implementations include R3 Corda, IBM Hyperledger Fabric and Digital Asset Platform.

As the research and development of the technology progresses, real-world applications have highlighted some of the challenges associated with these use-cases, including the need for safe, secure and scalable systems. \\

As of 2018, the DLTs impact in the financial sector seems still circumvented. According to a recent survey from \cite{multifundssurvey}, big data analytics, AI and robo-advice are currently higher up in the Asset Managers' agenda than DLT, explaining that it is "probably down to the fact that these new technologies are relatively easier to implement than more revolutionary DLT concepts". Despite strong progress in the research, it would seem that in the near-to-medium term many of the benefits and efficiency gains of DLT are likely to be reaped by start-ups and financial institutions in the developing countries, such as \cite{abracompany}, \cite{ripplecompany}, \cite{chaincompany}, \cite{bitpesacompany} and \cite{shocardcompany}. \\


\begin{figure}[h]
    \begin{tcolorbox}[colframe=boxcolor]
        \section*{What is a blockchain?}
        The term blockchain refers to the most well-known arrangement of DLTs (due to Bitcoin's fame), and refers to a distributed ledger architecture where the data is stored in entities called transaction blocks, linked with each other through chained encryption. The blockchain itself is the data structure formed by these linked blocks. Blockchains make use of algorithmic methods and encryption to ensure immutability, security and truthfulness of the information.
        
        New additions are initiated by one of the nodes that creates a new block of data, containing the encrypted transactions. Information about the block is then shared on the network, and all partecipants collectively try to determine the block's validity according to a pre-defined alghoritmic validation method (consensus). After the validation, all the partecipant can add the block to their copy of the ledger. With this mechanism, every change to the ledger is replicated across the entire network, and each node has a full, identical copy of the entire ledger at any point in time. As the chain grows and new blocks are added, earlier blocks cannot be altered. \\

        The cryptocurrency aspect is what has made Bitcoin garn the most fame. The consensus process involves the generation of a reward to the node that validated the last block of the blockchain, that being the currency in itself.
    \end{tcolorbox}
\end{figure}
\newpage

\section{Technical design elements}

It is emphasized that DLT is not a single, well-defined technology, but as of today there is a plurality of blockchains and Distributed ledgers (henceforth DLs) in active development.

DLs can be designed in a number of ways pertaining to main idea behind them and the use-cases they're designed to respond to. Such arrangements usually involve several key technical design concepts that specify how the information has to be kept on the ledger and how the latter has to be updated. 
There usually are four core attributes of DLTs, these are:
\begin{enumerate}
    \item The distributed nature of the ledger
    \item The consensus mechanism
    \item The cryptographic mechanisms
    \item The network entry permission level
\end{enumerate}

These three elements play are fundamental in ensuring the DLs ability to store and exchange data across different, self-interested parties, without the need for a central record-keeper, without the need for trust amongst the concerned parties, as it is guaranteed by the system itself, and while assuring that no double-spending takes place.

\subsection{Distributed nature of the ledger}
In its simplest form, a distributed ledger is a data store held and updated by each participant (or node) in a network. The control over the ledger does not lie with any one entity, but with several, if not all the network partecipants. This sets the technology apart from cloud computing or data replication, which are commonly used as shared ledgers.

No single entity of the network can amend past data entries, and no single entity can approve new additions to the ledger, which have to go through a predefined consensus mechanism. At any point in time there exists only one version of the ledger, and each network partecipant owns a full and up-to-date copy of it.
After validation the new transaction(s) are added to all the ledgers to ensure data consistency across the network.

This distributed nature of DLs allows the removal of a trusted central party, increasing speed and potentially removing friction costs and inefficiencies associated with the matching and reconiculiation processes. It also improves security, removing the single point of attack and single point of failure that is represented by the central trusted entity. To potentially gain control over the network, a malicious third party would have to gain control over 50\%+1 nodes in the network. 

Security risks aren't completely solved: the software layer built over the DL can become additional attack surfaces.

\subsection{Consensus mechanism}
\label{sec:consensus}
The purpose of the consensus mechanism is to verify that the information being added to the DL is legitimate. It is also important to handle conflicts between multiple simultaneous competing entries (ie double spending). There exist many different consensus algorithms or mechanisms, depending on the need and idea behind them, each has their advantages and disadvantages. \\

The Bitcoin blockchain uses Proof-of-Work (PoF) to enstablish consensus. To add a new block to the blockchain, a node has to provide a proof of work. This is a computationally taxing problem, but easy to verify, and is solved by brute-forcing cyptographic hashing algorithms until the correct string that satisfies certain conditions is generated. This process is called "mining". Each miner that produces a valid PoF is then rewarded Bitcoins, which serves as an economic incentive to maintain system integrity. \\

The Ethereum blockchain uses Proof-of-Stake (PoF). Its process is quite different from the PoF of Bitcoin, as there's no mathematical problem to solve, but instead, the creator of the new block is chosen in a deterministic way based on their stake, that is, how many coins or tokens they possess. 
A key advantage to this approach is the energy efficiency. The Bitcoin network, for example, requires an annual energy consumption comparable to that of Columbia (56.6 TWh annually). 
Thus PoS systems are well suited to platforms where there is a static coin supply, without inflation from block rewards. The rewards consist only in the transaction fees. \\

\begin{figure}[h]
    \centering
    \includegraphics[scale=0.3]{bitcoin-energy.png}
    \caption{
        Data as at April 3rd, 2018. Retrieved from https://digiconomist.net/bitcoin-energy-consumption}
\end{figure}

The Corda distributed ledger, utilizes a unique "pluggable" consensus service, due to its peculiar distributed ledger architecture. It divides consensus in two types, validity consensus and uniqueness consensus. Given a proposed transaction, the first type of consensus checks whether the two parties satisfy a set of validity conditions, going back through all the parties' transaction histories, while the latter has the purpose to avoid double spend, and the consensus is provided by a Corda network service called "Notary", by attesting that, for a given transaction, there hasn't been proposed another competing one.\\

\subsection{Cryptographic mechanisms}

Cryptography is at the core of the DLT. Each data entry is hashed, producing the so-called digest, which looks random and unrelated to the original input, but is in fact deterministic, meaning that from one original input there's only one hash possible. The data is hence hidden to anyone that is not intended to look at the data. Most DLTs use asymmetric cryptography for digital signatures, which is a common and robust method used in a wide array of application. Each network partecipat has a private key, that is used for signing digital messages and only known to the key proprietary, and a public key, which is public knowledge and used for validating the identity of the sender of the original message.

In the blockchain subset of DLs in particular, encryption plays a fundamental role, as they're essential in the chain encryption mechanism between the blocks that make up the blockchain itself.

\subsection{Network entry permission level}

The partecipation in the DL network can be open (permissionless) or permissioned. Bitcoin and Ethereum are the most prominent examples of completely permissionless blockchain, where partecipants can join or leave the network at will. This plays as one of their strengths, as a large, open permissionless system with a large number of nodes incentivized to validate new changes to the ledger and accurately and enstablishing a consensus is directly related to its network security (\ref{sec:consensus}).

In permissioned DLs its members are pre-selected by somone - an owner, or a network service - who controls network access and sets the rule of the ledger. 


% This solves problems with governments and regulators with concerns blabla about the identity verifications legal ownerships but disadvantage there's now a centralized trust entity playing a coordinating role (but after the entry everything is normal)

%Permissioned DLs do not typically require difficult “proofs of work” as a consensus mechanism for verifying transactions because network participantsare pre-selected and trusted