\chapter{Business analysis}
\label{chap:business-analysis}

\section{Industry growth}

DLT is still at an early stage of development. Challenges related to privacy, security, interoperability, regulatory and legal haven't been completely cleared as of yet, and there hasn't been yet a solid technology taking a strong position in the market.

\subsection{Governments stance}

The UK Government's Office of Science published a major report on DLT in January 2016, \cite{ukgovdltpaper}, assessing the possibilities of DLT use in private and public areas. The Department for Work and Pensions in the UK has been testing from June 2016 the use of DLT for welfare benefit payments, that trough a phone application lets welfare claimants manage their benefit money, having transaction recorded on a DL with the aim to create a more solid and efficient welfare infrastructure preventing frauds.

The Estonian government has been experimenting with DLTs for a long time, as apparent from its \cite{eresidency} platform, where it's possible to verify government records like birth or marriage certificates, which are only a couple of the services provided by, as the e-Residency project seem to be trying to encompass a whole variety of utilities, like opening a bank account or starting a company (in Estonia), but most importantly, providing a form of transnational digital identity, so far as to having NASDAQ partnering with the platform to enable secure e-voting in shareholders meetings.

TODO\\

\subsection{Financial stance}

The two biggest trends in DLT development seem to be:
\begin{enumerate}
    \item Commercial FinTech startups developing applications for different purposes utilizing public blockchain infrastructure, like BitCoin and Ethereum.
    \item Industry consortiums researching and developing private, permissioned distributed ledgers to address a set of industry-specific enterprise use-cases.
\end{enumerate}

A survey by Greenwitch Associates, \cite{greenwitchdltreport}, gathering 400 market partecipants (as in financial companies, funds, etc) working on DLT has tried to assess opinions on the key trends and issues in the current state of DLT development. The report finds engagement across the sample at 63\%, with Market Infrastructure providrs in the high end with 75\% engagement, and Asset Managers in the bottom, with 32\% engagement.

This presents a picture of lagging in the adoption of the technologies, due to the competition of other competing technologies as mentioned before. 

But as \cite{ibmdltreport} reports, recent evidence shows that the industry is operating in the research and development department. 
Examples (extrapolated from the report), include:

\begin{itemize}
    \item Schroders announcement that they have joined the Hyperledger Project
    \item Northern Trust’s implementation of a blockchain platform for Private Equity fund administration in partnership with Unigestion and IBM
    \item BlackRock’s announcement of their intention to Blockchain-enable Provider Aladdin, a private platform / dashboard to streamline transactions with BlackRock’s Custodians
    \item The launch of a Blockchain-based solution for syndicated loan servicing by Synaps (a joint venture of Ipreo and Symbiont), with involvement from Asset Managers including Eaton Vance and Alliance Bernstein
    \item Calastone’s launch of Blockchain-based distributed market infrastructure
    \item FNZ ‘s development of FNZChain as a private blockchain for Asset Management registers
    \item SETL’s launch of Iznes, in collaboration with various asset managers, as a Pan-European Distribution and Transfer Agent Platform for fund subscriptions, distributions and
    settlements
    \item The announcement from Natixis in July 2017 that they had successfully sold funds directly to clients through FundsDLT, a fund distribution platform developed by a partnership of the Luxembourg Stock Exchange / Fundsquare, InTech and KPMG;
    \item The news that SEB are working with NASDAQ on the development of a trading platform for Swedish mutual funds
    \item The rise of crypto fund launches in 2017, representing the fastest growth of any hedge fund sector in the industry’s history, \cite{hedgefundcryptoreport}
\end{itemize}

TODO\\

The technology has the potential to support attractive models for the tech-savvy generation of consumers who wants more control over their investments and finances, delivered via their favoured media.