\addcontentsline{toc}{chapter}{Abstract}

\newpage

\mbox{}\vfill
\chapter*{\centering Abstract}

Blockchains have been a disrupting force in the financial market. Popularized by Bitcoin, blockchains, and its underlying technology, Distributed Ledger Technology (DLT), has shaped up to be much more than the foundation for cryptocurrencies.\\

DLT can have applications in cross-border payments, or financial market infrastructures but its potential is not only limited to the financial sector, as it can enable digital identity solutions, or tamper-proof decentralized records for the flow of any kind of good, service or transaction.
More generally, this technology can enable more secure, resilient and efficient systems, making it possible for further decentralization, deintermediation, greater transparency and cost reductions.\\

However, the technology is still in its infancy and it's rapidly evolving. This coupled with the cost of the migration for existiting longstanding IT infrastructures to a DLT infrastructure poses many new risks and challenges.

The aim of this thesis is to provide an analysis of Distributed Ledger Technologies, with the objective of identifying their risks, opportunities and implementation viability on different scales. 

The research work starts from a brief introduction to the technology and how the industry has been adapting to it, with statistical and analytical evidence presented. The thesis then develops on the comparative analysis between the main competing technologies.

In accordance to the research findings, the project also aimed to provide a proof-of-concept design of a solution for a prototypical system answering to at least one identified use-case using the most suited DLT implementation between the analyzed ones.

\mbox{}\vfill
