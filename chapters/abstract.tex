\addcontentsline{toc}{chapter}{Abstract}

\newpage

\mbox{}\vfill
\chapter*{\centering Abstract}

Bitcoin and blockchains have been a disrupting force in the financial market. While these have a certain standing of their own - through cryptocurrencies - the underlying technology, Distributed Ledger Technology, is shaping up to transform the financial services sector. 
In most financial contexts, financial infrastructures are trusted by the counterpart(s) of a transaction with maintaining, updating and preserving the integrity of the data in a central ledger, in addition to managing certain risks on behalf of the counterparts.

The financial industry has been exploring different ways of leveraging the technology to reducing inefficiencies springing from this centralized mode, such as (i) reduction of transactional frictions, (ii) maintenance of accurate record-keeping between infrastructures,  (iii) reduction of infrastructure complexity, (iv) increase in end-to-end processing (that is, the speed in the obtaining and availability of assets and funds), (v) better management of operational and financial risks. 


\iffalse 

DLT could reduce the traditional reliance on a central ledger managed by a trusted
entity for holding and transferring funds and other financial assets.
transparency and immutability in transaction record-keeping?
improving resilience through distributed data management?
\fi
\mbox{}\vfill
