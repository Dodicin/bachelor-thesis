\addcontentsline{toc}{chapter}{Objective}

\newpage

\mbox{}\vfill
\chapter*{\centering Objective}

The aim of this project was to provide an analysis to Distributed Ledger Technologies, with the objective of identifying their risks, opportunities and implementation viability on different scales. The research work starts from a brief introduction to the technology and how the industry has been adapting to it, with statistical and analytical evidence. The project presents then a brief overview of three main DLT platforms, with considerations on which technology to adopt, and why.

In accordance to the research findings, the project also aimed to provide a proof-of-concept architectural design of a solution for a credits interchange system using the most suited DLT implementation between the analyzed ones.

\section*{Structure of the thesis}

\begin{itemize}

    \item In the \nameref{chap:introduction} chapter a general introduction to the Distributed Ledger Technology is given, expanding on its technical design elements.

    \item In the \nameref{chap:business-analysis} chapter, data on the the actual industry growth of the technology is presented, with statistical data and analysis from different sources.

    \item In the \nameref{chap:comparative-analysis} chapter, an overview and general comparison of the features of the three DLT technologies is given.

    \item In the \nameref{chap:corda-analysis} chapter, a primer on how Corda is structured is given, to better understand the choices of the prototypical implementation.

    \item In the \nameref{chap:prototype-design} chapter, a prototype design is presented, showcasing how a DLT solution can automatize and implement a seamless distributed voucher payment system.

    \item In the \nameref{chap:conclusions} chapter, the author draws the conclusions on the results of the project.


\end{itemize}

\mbox{}\vfill