\chapter{Conclusions}
\label{chap:conclusions}

The focus of this project work has been the exploration and viability research of Distributed Ledger Technology in the current market environment, to understand if a solution can carve itself a place in the market, can be profitable, how the landscape of DLTs is formed, and in what form could such a solution be presented. 

We started from a general primer on the technology itself and came to understand what position in the market the technology currently holds, and how the major market players have been reacting to it. 
Evidence has shown a moderate interest in the technology, but with a growing engagement and flux of investments that is expected to grow exponentially in the next years.

After that a comparative analysis between three main technologies, Ethereum, Hyperledger and Corda revealed their major characteristics and difference in approach, showcasing which technology is more adequate for which kind of solution. 

Corda emerged as the preferred technology for a distributed solution with strong focus on financial use-cases that require a certain degree of privacy, regulations and security.

Finally, we evaluated a possible solution design and its implementation, focusing on the usage of vouchers in transactions. 
In this, the issues about leftover change for voucher usage have been addressed by two different solutions.
This design has shown how a distributed ledger architecture can be implemented to enable a seamless payment platform, automatizating every transaction and data recording thereof through distributed ledger technology. It has also shown how such an architecture can solve the issues of cost, latency, single-point-of-failure, eventual reconciliation and need for a trusted intermediary for a payment use case.