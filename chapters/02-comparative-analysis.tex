\chapter{Comparative Analysis}
\label{chap:comparative-analysis}

This chapter aims to analyze three major DLTs: Ethereum, Corda and Hyperledger. It will provide an overview of each DLT, and then dive into a comparison between the technical aspects of each, in order to showcase the major differences among them.

\section{Ethereum overview}

Ethereum has one of the biggest following among users and developers, with more than a thousand applications already built on top of the blockchain. It was developed as a permissionless public blockchain, where anyone can build application or write smart contracts using its own programming language, Solidity. Ethereum's development team's goal was to provide a generic toolbox for providing support to a wide range of decentralized applications, with a particular emphasis on situations where rapid development time, security and of different applications to efficiently interact are important. \\

The structure of the Ethereum blockchain is very similar to Bitcoin's. Every node on the network stores a copy of the entire transaction history. With Ethereum, every node also stores the most recent state of each smart contract, in addition to all the transactions. 


\section{R3 Corda overview}
Corda is a DLT backed by R3, a consortium made up of many big financial institution. It was developed as a global ledger, with its main goal is to provide an architecture to enable frictionless, well-regulated, reliable and private agreements between parties. Corda was developed specifically on financial use-cases, its main field of application being the financial services industry.

\section{Hyperledger Fabric overview}