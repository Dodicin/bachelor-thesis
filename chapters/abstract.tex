\addcontentsline{toc}{chapter}{Abstract}

\newpage

\mbox{}\vfill
\chapter*{\centering Abstract}

Bitcoin and blockchains have been a disrupting force in the financial market. While these have a certain standing of their own - mainly through cryptocurrencies - the underlying technology, Distributed Ledger Technology (hereon DLT), is shaping up to transform the financial services sector. 
In most financial contexts, financial infrastructures are trusted by the counterparts of a transaction with maintaining, updating and preserving the integrity of the data in a central ledger, in addition to managing certain risks on behalf of the counterparts.
This centralized model carries with it numerous inefficiencies, such as, to name a few: (i) transactional frictions, (ii) maintenance of record-keeping between different infrastructures (e.g. banks),  (iii) infrastructure complexity, (iv) end-to-end processing slowness (that is, the speed in the obtaining and availability of assets and funds), (v) management of operational and financial risks. 
Thus, the financial and technology (FinTech) industry has been exploring different ways of leveraging the DLT to reduce the overhead and cost springing from the centralized model, increasing the drive for the research and development of this technology.



\iffalse 
// appunti
DLT could reduce the traditional reliance on a central ledger managed by a trusted
entity for holding and transferring funds and other financial assets.
transparency and immutability in transaction record-keeping?
improving resilience through distributed data management?
\fi
\mbox{}\vfill
